%***********************************************************************
%
%	$Id: manual.tex,v 1.62 2023/04/04 03:49:54 warme Exp $
%
%	File:	manual.tex
%	Rev:	e-6
%	Date:	04/03/2023
%
%	Copyright (c) 2002, 2023 by David M. Warme, Pawel Winter and
%	Martin Zachariasen.  This work is licensed under a Creative
%	Commons Attribution-NonCommercial 4.0 International License.
%
%***********************************************************************
%
%	LaTex source code for the GeoSteiner manual.
%
%***********************************************************************
%
%	Modification Log:
%
%	c-1:	05/04/2007	warme
%		: Added proper banner and mod log.
%		: Updated copyright notices to 2007.
%		: Replaced "publicly available" clause with one that
%		:  says "version 3.1 still available from ..., but is
%		:  unsupported by GeoSteiner, Inc."
%		: Commented out the paragraph claiming that (i) the
%		:  commercial version is any better than the academic
%		:  version, and (ii) that empirical data can be
%		:  obtained from the DIKU web site.
%		: Computational state of the art as of 5/2007.
%		: Removed all references and discussion of the
%		:  source code.
%	c-2:	02/02/2014	warme
%		: Replace obsolete image-inclusion with newer
%		:  technology.
%	e-1:	04/14/2015	warme
%		: Changes for 5.0 release.
%	e-2:	09/05/2016	warme
%		: Changes for 5.1 release.
%		: Changes to rand_points program.
%		: Document changes to fst2graph arguments and
%		:  SAVE_FORMAT parameter.
%		: Change notices for 5.1 release.
%	e-3:	01/01/2017	warme
%		: Update dates for 5.1 release.
%	e-4:	01/01/2017	warme
%		: Fix two syntax errors.
%	e-5:	11/18/2022	warme
%		: Corrected documented default value of
%		:  max-cutset-enumerate-comps.
%		: Added new parameters: sparse-subtours,
%		:  zero-weight-cutsets-method, strengthen-reduce,
%		:  generate-unstrengthened, initial-primal-heuristic,
%		:  ib-node-limit, bb-lp-limit,
%		:  initial-primal-heur-stop.
%	e-6:	04/03/2023	warme
%		: Version 5.3.
%		: Document new callback interfaces.
%		: Document our enabling of floating-point exceptions.
%		: Added local cuts trace style.
%
%***********************************************************************

\documentclass[12pt,twoside,letterpaper]{article}
\usepackage{times}
%\usepackage{amsfonts}
%\usepackage{mathptm}
% \usepackage{psfig}
\usepackage{epsfig}
\usepackage{makeidx}
\usepackage{verbatim}

\def\func#1{\bigskip\noindent{\bf\large #1}\\}
\def\comment#1{\bigskip\centerline{\parbox{11cm}{\small #1}}\bigskip}
\def\code#1{{\ttfamily #1}}
%\def\geosteiner{{\sc Geo\-Steiner}}
\def\_{\leavevmode\kern.04em\vbox{\hrule width 0.4em height 0.3pt}}
\def\geosteiner{Geo\-Steiner}
\makeindex

\newcommand{\GeoSteinerCurrentVersion}{5.3}

\begin{document}
\thispagestyle{empty}
\vspace*{4cm}

\centerline{\Huge\bf \geosteiner{} \GeoSteinerCurrentVersion{}}

\vspace*{3cm}

\centerline{\Large User's Guide and Reference Manual}

\vspace*{10cm}
\centerline{Copyright \copyright{} 2004, 2022 by
 David~M.~Warme, Pawel~Winter and Martin~Zachariasen.}
\centerline{This work is licenced under a Creative Commons}
\centerline{Attribution-NonCommercial 4.0 International License.}

\clearpage
\thispagestyle{empty}
~

\clearpage
\pagenumbering{roman}
\tableofcontents

\clearpage
\setlength{\parskip}{1.5ex plus0.1ex minus0.1ex}
\setlength{\parindent}{0ex}

\pagestyle{headings}
\section*{Preface}

This manual documents \geosteiner{} version
\GeoSteinerCurrentVersion{}
--- an optimization software package for solving Steiner tree
problems.
\geosteiner{} version 4.0 was a proprietary commercial product,
that was released in substantially identical form under an open source
form beginning with version 5.0.
\geosteiner{} version 3.1 is still available from
\code{www.geosteiner.com} under an academic license, but is no
longer supported.

Version \GeoSteinerCurrentVersion{} contains significant improvements
over the previous
version (\geosteiner~3.1); these improvements are both functional and
structural.
By far the largest structural change is that the core
optimization algorithms have now been encapsulated into a library of
callable subroutines.  This greatly facilitates the incorporation of
\geosteiner{} into other applications.  Indeed, the old familiar
stand-alone programs from version 3.1 have now been completely
re-implemented to use only the documented library interfaces.  The
ability to use function calls instead of program calls from
applications provides for more efficient solution of (large) series of
problem instances; such applications occur frequently in, e.g., VLSI
layout.  In addition, the library interfaces provide greater control
of the solution process when needed.

% There is a commercial version of \geosteiner{} that offers additional
% functionality, and significantly improved running times for many
% problem instances.  Refer to \code{www.diku.dk/geosteiner} for
% empirical data comparing the relative performance of the public and
% commercial versions of \geosteiner.  Any commercial use of the
% software requires a license; please consult the LICENSE file for
% additional information.

The authors would like to thank Benny K.\ Nielsen, who has been the
programmer on the callable library project. In addition, Benny has
written a major part of the new FST generator for uniformly-oriented
Steiner trees.

\bigskip\bigskip\bigskip

\begin{flushright}
Copenhagen/Washington, April 2023

\bigskip\bigskip
David M.\ Warme\\
Pawel Winter\\
Martin Zachariasen

\end{flushright}

\clearpage
\thispagestyle{empty}
~

\clearpage
\pagenumbering{arabic}
\section{Introduction}
\label{intro}

\geosteiner{} is a software package for solving Steiner tree
problems. The package currently solves the following NP-hard problems
in the plane\footnote{Problem definitions are given in Section~\ref{intro_stp}.}:  
\begin{itemize}
\item Euclidean Steiner tree problem.
\item Rectilinear Steiner tree problem.
\item Uniformly-oriented Steiner tree problem (including hexagonal
  and octilinear Steiner tree problems).
\end{itemize}
In addition, the package gives the user access to a powerful solver
for 
\begin{itemize}
\item Minimum spanning tree in hypergraph (MSTHG) problem.
\end{itemize}
The solver for this NP-hard problem is used as a subroutine in the
solution of the above geometric problems. 

\geosteiner{} is written in ANSI C. The code makes heavy use of
linear programming 
(LP); the public domain LP-solver \code{lp\_solve} is included (in a
significantly modified form). However, the package also supports
\code{CPLEX}, a proprietary product of IBM Inc., which is perhaps
the fastest and most robust LP-solver available. 
The authors of \geosteiner\ strongly recommend that you use CPLEX if at
all possible --- especially for production applications or published
computational studies. The core callable library requires no
supplementary software or libraries (except the CPLEX library if
\geosteiner{} is configured to use CPLEX as its LP solver). 

% \comment{DMW: The phrase ``requires no supplementary software or
%   libraries,'' is problematic.  Consider TeX, X11, QT, and now even
%   eps2eps, which is very hard to get on my Linux here at home. MZ: I
%   have change the paragraph above so that we only refer to the core
%   callable library. Okay?} 

In this introductory section we first define the problems that are
solved by \geosteiner, and give some fundamental definitions and
acronyms used throughout this manual (Section~\ref{intro_stp}). Then an
introduction to the callable library and associated stand-alone
problems is given (Sections~\ref{intro_callable}
and~\ref{intro_standalone}). Finally, we give some historic background
on \geosteiner{} in Section~\ref{intro_literature}.


\subsection{Steiner tree problems}
\label{intro_stp}

Given a metric and a (finite) set of points in the plane, also denoted
terminals, a {\em Steiner minimum tree} (SMT) is a shortest possible
interconnection of the terminals. This interconnection must be a tree,
and may contain junctions 
that are not among the terminals, so-called {\em Steiner points}. In
Figure~\ref{fig:SMTs} we show four different SMTs for the same set of
terminals. These are SMTs under the Euclidean, rectilinear, hexagonal
and octilinear metric, respectively. 

\begin{figure}[tbp]

\begin{tabular}{cc}
\includegraphics[width=6.5cm,clip=]{fig_ESMT.eps} &
\includegraphics[width=6.5cm,clip=]{fig_RSMT.eps} \\
\includegraphics[width=6.5cm,clip=]{fig_HSMT.eps} &
\includegraphics[width=6.5cm,clip=]{fig_OSMT.eps} \\
\end{tabular}

\caption{SMTs for the same set of terminals under different metrics
(output from \geosteiner).}
\label{fig:SMTs}
\end{figure}

All metrics currently handled by \geosteiner{} are {\em uniformly
oriented metrics}: Given a set of $\lambda \geq 2$ uniformly oriented
directions in the plane, the distance between two points is defined to
be the length of a shortest path in which all line segments have one
of the given directions. As special cases we have the rectilinear
($\lambda=2$), hexagonal ($\lambda=3$), octilinear ($\lambda=4$) and
Euclidean ($\lambda=\infty$) metric.

If we break an SMT at all terminals having two or more incident
edges, each component will be a so-called {\em full Steiner tree}
(FST). These are trees in which all terminals are leaves. The fact
that the number of terminals in each FST in an SMT usually is small
is what makes it possible to solve large problem instances to
optimality. More specifically, the algorithms employed by
\geosteiner{} first generate a set of FSTs known to contain an SMT as
a subset, and then the shortest possible union of FSTs interconnecting
all terminals is selected; 
we say that the FSTs are concatenated. The concatenation problem is
equivalent to finding a {\em minimum spanning tree in a hypergraph}
(problem MSTHG).
An efficient solver for this subproblem forms a cornerstone of
\geosteiner.


\subsection{Callable library}
\label{intro_callable}

The kernel of \geosteiner\ is the callable library. Both high-level
and low-level interfaces are provided.  Also included are powerful
routines for manipulating various algorithmic parameters, handling
messages, and accessing problem instance data in various formats.

The high-level interfaces give the user easy access to the basic
algorithms in the library. Problem instances are given as simple
arrays, and the functions return optimal solutions to the problem 
instances. 

Low-level interfaces are provided for users who need more control
over the solution process. Also, the low-level
interfaces are used by the stand-alone programs that accompany the
callable library. For more details on the design and structure of the
callable library, please consult the user's guide in
Section~\ref{usersguide} and the callable library reference manual in
Section~\ref{libraryfunctions}.


\subsection{Stand-alone programs}
\label{intro_standalone}

The stand-alone programs are provided for users who would like to
solve Steiner tree problems without writing their own application
programs. For example, if the coordinates of the given problem
instance are given in a file, the stand-alone programs give the user
the opportunity to solve the instance and make a postscript plot of
the solution. A complete list of all stand-alone programs, including
docu\-mentation of their invocation options, and examples of their use
are given in Section~\ref{standalone}.


\subsection{Historic note and literature}
\label{intro_literature}

To the best knowledge of the authors, as of April 2023, GeoSteiner
represents the computational state of the art for geometric Steiner
tree problems in the plane under each of the following metrics:
\begin{itemize}
 \item Euclidean
 \item Rectilinear
 \item Uniformly oriented metrics
\end{itemize}
Furthermore, GeoSteiner has held this dominant position continuously
since at least 1998.
During the $11^{\mathrm{th}}$ DIMACS Implementation Challenge
(December, 2015) no other algorithms world-wide were entered into any
of these problem categories.
(Because GeoSteiner was only entrant in each of these categories, no
competition was performed --- which is why GeoSteiner does not appear
in any of the official DIMACS 11 competition results.)

The ``GeoSteiner'' name was coined (and is therefore ``owned'') by Pawel
Winter, whose seminal program GEOSTEINER started it all back in
1985~\cite{Winter85}. In 1996 Winter and Zachariasen published an
improved algorithm called ``GeoSteiner96''~\cite{WinterZachariasen97}.

On the other hand, Warme's first Steiner tree code was the Salowe-Warme
algorithm in 1993, which used backtrack search to concatenate
rectilinear FSTs~\cite{SaloweWarme95}.  In 1998, Warme's Ph.D.
dissertation~\cite{Warme98} described a new branch-and-cut code for
finding minimum spanning trees in arbitrary hypergraphs --- which was
applied to the FST concatenation problem for both rectilinear and
Euclidean FSTs. 

The first distribution of the combined code therefore represented the
``third version'' of each group's code, and it was thus named
\geosteiner\ version 3.0. This and subsequent versions continue
that naming convention.

The algorithms in \geosteiner~3.0 are based on those described
in~\cite{WarmeWinterZachariasen00, WinterZachariasen97,
  Zachariasen99a}.

\geosteiner~4.0 was a proprietary commercial product which introduced
the callable library interfaces, and support for solving
uniformly-oriented Steiner
trees~\cite{NielsenWinterZachariasen02c}.
In addition, a number of minor improvements were made throughout the
code.

Upon termination of commercial operations in 2015, the GeoSteiner code
was released once again in open source form as \geosteiner{} version
5.0, and its various successors.


\clearpage
\section{Callable Library User's Guide}
\label{usersguide}

In this section we give a number of examples of using the callable
library. We start with a few simple uses of the high-level functions,
then move to the low-level interfaces, and finally, we discuss the use
of callback functions. 

\subsection{High-level interfaces}
\label{high_level_interfaces}

Any application program that uses the \geosteiner{} library must include
the \geosteiner{} header file \code{geosteiner.h}. Furthermore, 
the \geosteiner{} environment must be opened using the function {\bf
gst\_open\_geosteiner()} as described in
Section~\ref{environment_functions} on
page~\pageref{environment_functions}.

Our first example, shown in Figure~\ref{fig:demo1}, computes an Euclidean SMT
for the points $(0,0)$, $(0,1)$, $(1,0)$ and $(1,1)$.  
After having successfully opened the \geosteiner{} environment, we
use the high-level function {\bf gst\_esmt()} to compute the SMT (see
page~\pageref{gst_esmt}). As arguments we first pass the number of
terminals, here~4, and then a double array {\bf terms} that holds the
terminal point coordinates. Then follow the variables {\bf length},
{\bf nsps} and {\bf sps}, in which the length of the computed SMT,
the number of Steiner points and the coordinates of the Steiner
points will be returned.  The remaining arguments to {\bf gst\_esmt()}
are all given as NULL, causing corresponding inputs to assume default
values, and corresponding outputs to be ignored; in particular the
edges of the optimal solution and the solution status are all ignored,
and default values are assumed for all GeoSteiner parameters.

In the program we assume that {\bf gst\_esmt()} returns successfully
--- the return code is not checked --- and then we print the length of
the SMT and the (two) Steiner points. Finally, we close the
\geosteiner{} environment and the program ends. We encourage you to
run the \code{demo1} program that comes with the \geosteiner{}
distribution.

\begin{figure}[p]
{\scriptsize
\verbatiminput{demo1.notabs}
}
\caption{Demo program that computes an Euclidean SMT for four
terminals (demo1.c)}
\label{fig:demo1}
\end{figure}

Our next example, \code{demo2}, computes a series of SMTs for randomly
generated points sets (Figure~\ref{fig:demo2}). This program has
two command line parameters: Firstly, the $\lambda$-value for the uniformly
oriented metric that is to be used --- where $\lambda=0$ is defined to
be the Euclidean metric; secondly, the required maximum excess from
optimum in percent. As an example, ``\code{demo2 4 1}'' computes Steiner trees
using the octilinear metric whose length are at most $1\%$ from optimum. 

In the program we first read the command line parameters and then
create a metric object (see Section~\ref{metric_functions}) that
corresponds to the given command line para\-meter. Then we create a
default parameter set (see Section~\ref{parameter_functions}) and
change the \code{GST\_PARAM\_GAP\_TARGET} parameter (see
Appendix~\ref{parameters}). Finally, we compute the SMTs using the
high-level function {\bf gst\_smt()} which takes the metric object and
parameter set as arguments. Only the SMT length is returned from {\bf
  gst\_smt()}, and based on this the total length of all SMTs is
computed and displayed. 

\begin{figure}[p]
{\scriptsize 
\verbatiminput{demo2.notabs}
}
\caption{Demo program that computes SMTs for a series of randomly
generated problem instances (demo2.c).}
\label{fig:demo2}
\end{figure}

Our third example, shown in Figure~\ref{fig:demo3}, is similar
to the previous example, but instead of generating the terminal
coordinates using a random number generator, we read the terminal
coordinates from input. The input is assumed to 
be in the OR-library format\footnote{OR-library:
  http://www.ms.ic.ac.uk/info.html}. The program reads every instance 
in the file and computes an SMT for each. The metric is given
as the first command line parameter to the program. Furthermore, the
maximum FST size (number of terminals) can be specified as the second
command line parameter. By giving (small) bound on the FST size, the
running time of FST generation may decrease significantly --- at the
cost of not necessarily returning the optimal solution.

\begin{figure}[p]
{\scriptsize 
\verbatiminput{demo3.notabs}
}
\caption{Demo program that computes SMTs for a series of instances
read from an OR-library file (demo3.c).}
\label{fig:demo3}
\end{figure}


\subsection{Low-level interfaces}
\label{low_level_interfaces}

The low-level interfaces completely separate FST generation and FST
concatenation, the two components of the exact algorithm used by
\geosteiner{}. Thus it is possible to use alternative FST generation
or concatenation algorithms --- or to store away generated FSTs and
concatenate them at a later time.

Another advantage of using the low-level interfaces is the greater
control they provide over the solution process, in particular with
respect to solving the FST concatenation problem. For most large
instances the FST concatenation problem --- which is equivalent to
solving a MSTHG problem --- is by far the most time-consuming part of
the solution process.

As for the high-level interfaces, programs that use the low-level
interfaces must include the \geosteiner{} header file and open
the \geosteiner{} environment. In the example given in
Figure~\ref{fig:demo4} we construct a large random terminal set,
generate the rectilinear FSTs, and set up a solution state object for
the MSTHG problem (see Section~\ref{solver_functions}). One of
the parameters passed to the solution state object is
\code{GST\_PARAM\_CPU\_TIME\_LIMIT}, which limits the amount of time
spent in the solver before returning to the application program. 

\begin{figure}[p]
{\scriptsize
\verbatiminput{demo4.notabs}
}
\caption{Demo program that computes a rectilinear Steiner tree (not
necessarily minimal) for a large random terminal set. The upper
bound/lower bound gap is displayed at fixed running time intervals
(demo4.c).} 
\label{fig:demo4}
\end{figure}

In the main loop of the program we run the MSTHG solver by calling {\bf
gst\_hg\_solve()}, passing the solution state object as an
argument. When this function returns, we query the solution state
object for the current solution status; this is done by calling {\bf
gst\_get\_solver\_status()} which returns a code that represents the
four possibilities (optimal solution, feasible solution, infeasible
problem, no feasible solution yet).

If a feasible solution has been found, the current upper and lower
bound is obtained by querying the solution state property list. In our
example we repeat the main loop until we have found a feasible
solution that is within the maximum specified excess from optimum.


\subsection{Algorithmic callback functions}
\label{callback_interfaces}

Algorithmic callback functions provide the lowest-level --- and
perhaps most powerful --- of all the interfaces in the \geosteiner{}
callable library.
Callbacks are user-written functions.
Such functions become callbacks by passing their address to suitable
\geosteiner{} routines.
Once a function is established as a callback in this manner,
\geosteiner{} automatically invokes the function at the corresponding
critical points in the branch-and-cut algorithm.
For example, the user can provide callback routines that are invoked
every time
\begin{itemize}
  \item an LP is solved during the optimize / separate loop,
  \item processing of a branch-and-bound node completes,
%  \item a new lower bound is obtained,
  \item a new upper bound is obtained.
%  \item a branch variable is selected.
\end{itemize}
Callback functions permit the user to extend the \geosteiner{}
optimization algorithms by incorporating application specific
knowledge into some of \geosteiner{}'s most critical decisions.
As an example, the \code{bb} program (see Section~\ref{standalone})
uses callbacks to implement the 
\code{bb -r} switch: a callback function is defined that is invoked
upon completion of every node.  When invoked for the root node, the LP
solution is fractional, and the \code{-r} switch was specified, this
callback generates a postscript plot of the node's LP relaxation.
% Consult the \geosteiner\ source code (file \code{bbmain.c}) for an
% example of how this callback function is used.


\clearpage
\section{Callable Library Functions}
\label{libraryfunctions}
\def\adescr#1{\begin{minipage}{9cm}{\vspace{2pt}#1\vspace{2pt}}\end{minipage}}

\subsection{Application programming interface}
\label{sec:api}

All declarations needed to use the \geosteiner\ library in an
application are defined in a single include file called
\code{geosteiner.h}. All identifiers \code{\#define}'d in the
header file begin with the prefix ``\code{GST\_}''.  All structure or
union tags and typedefs begin with the prefix ``\code{gst\_}''.  All
functions provided by the library begin with the prefix
``\code{gst\_}''.

All \geosteiner\ library functions reside in a single library.
On most systems the name will be \code{libgeosteiner.a} and linking is 
done with \code{-lgeosteiner}. A shared library is also possible on 
some systems. If \geosteiner{} has been configured to use CPLEX as its
LP-solver, then the CPLEX callable library must also be linked with
the application program.


\subsection{Design of library}
\label{sec:lib_design}

The \geosteiner{} library is designed to be completely re-entrant so
that multiple problems can be solved serially or in a round-robin
fashion.  The current implementation might not yet completely satisfy
this goal --- especially concerning the various LP-solver interfaces
and the way \geosteiner{} interacts with them.  We hope to eventually
make the library fully thread-safe so that multiple problems can be
solved in parallel within a single process address space on a
multi-processor system.  However, this ideal is not yet supported in
the current version.
% Note that this is an ideal that may not be achievable due to
% limitations within CPLEX or within the particular C runtime
% library being used.  Whenever possible and reasonable, such
% limitations should arise from external sources --- not our own code.
%
% \comment{DMW: lp\_solve fails here!  CPLEX probably does also! MZ:
%  Yes. Could you please suggest an appropriate text?} 

All output generated by the library (i.e., text that was written to
stdout or stderr by previous versions of \geosteiner) is now
user-controllable.  Various types of output have parameters that
enable/disable their generation.  This is achieved using so-called
``channels'' described in Section~\ref{message_functions}. By default,
library routines are completely ``quiet''.

The library does not use any signals nor does it establish any signal
handlers.  These would be potential points of contention with
applications that use the library.  Instead, all asynchronous requests
to alter or abort a GeoSteiner computation (e.g., to abort the
solution process, force branching in lieu of constraint generation,
etc.) are delivered by a single routine that is designed to be safe
when called from a user-defined signal handler (see the description of 
the function {\bf gst\_deliver\_signals()} on
page~\pageref{gst_deliver_signals} for more information).  

During execution of its library functions, \geosteiner{} enables the
delivery of floating-point exceptions to the calling process/thread as
SIGFPE (or other) signals.
This is done for floating-point overflow, divide-by-zero and invalid
operation exceptions.
(By default on most systems, these floating-point exceptions generate
and propagate not-a-number (NaN) values.
Not all NaN values generated within an algorithm reach the outputs of
that algorithm, nor are NaN results always ``noticed'' by their callers.
To improve reliability, robustness, debug ability and trustworthiness
of its numeric results, \geosteiner{} forces these exceptions to
immediately signal their occurrence.
This is currently supported on the \code{x86\_64} and \code{arm64}
platforms.)

\subsection{Library objects}
\label{sec:lib_objects}

\subsubsection{\geosteiner\ environment}

The \geosteiner\ environment encapsulates licensing information and 
platform-specific data. If CPLEX is used as LP solver, the CPLEX
environment is stored here. 

The environment is a single {\em global} variable. No explicit user
references to the environment are possible, but the environment {\em
  must} be initialized by calling the {\bf gst\_open\_geosteiner()}
function (see Section~\ref{environment_functions}).


\subsubsection{Parameter set}

A parameter set holds values for all parameters used by the
library. In order to change one or more parameters, the user creates a
new parameter set and modifies the parameter(s) in this set. A pointer
to the parameter set (type \code{gst\_param\_ptr}) is then
passed to all functions for which these parameter settings should have
effect. Whenever a \geosteiner{} function accepts an argument of type
\code{gst\_param\_ptr}, the user may pass a NULL pointer in which case
the \geosteiner{} library assumes default settings for all parameter
values.

Parameter setting and querying functions are described in
Section~\ref{parameter_functions}, while the individual parameters are  
described in Appendix~\ref{parameters}.


\subsubsection{Problem instance}

The problem instance object is a {\em hypergraph} that can be
decorated with a variety of additional (and optional) data (see
Section~\ref{hypergraph_functions}). 
By attaching information globally to the hypergraph, and to its
vertices and edges, the problem to be solved becomes well-defined. In
general we would like to construct a {\em tree} in the hypergraph.
% note that the tree need not be a spanning tree. 
Problem instance objects have type \code{gst\_hg\_ptr}.

% For each vertex we have the following optional information:
% \begin{itemize}
% \item Vertex weight (default: zero)
% \item Is vertex a terminal? (default: yes)
% \item Which group(s) does the vertex belong to? (default: each
%   terminal vertex forms a singleton group)
% \end{itemize}
% A group is a specified subset of vertices; groups are allowed to
% overlap such that a given vertex can belong to more than one group. 
% Feasible (tree) solutions are constrained to include {\em at least}
% or {\em exactly} one vertex from each group, depending on the
% parameters given. If no group information is given, each terminal
% vertex for a singleton group.

% For each hyperedge we have the following optional information:
% \begin{itemize}
% \item Edge weight (default: 1)
% \end{itemize}

% A hypergraph may be embedded into $k$-dimensional space,
% $k \geq 2$ (default: no embedding). When the hypergraph is embedded,
% the following additional information is needed:

% \begin{itemize}
% \item Coordinates of vertices in $k$-dimensional space
% \item Geometric embedding of hyperedge; simply a list of line segments
%   giving the exact drawing of the hyperedge in $k$-dimensional
%   space 
% \item Metric (not needed)
% \end{itemize}

% \comment{MZ: As you can see I followed your suggestion of isolating
%   the embedding information completely. However, I have also chosen to
%   use a very general scheme to embed the hypergraph. This means, for
%   example, that a rectilinear edge that consists of two line segments
%   must be given as two lines segments in this specification. I think
%   this is much easier and equally powerful to what we already
%   have (and it simplifies the drawing routines tremendously). Also,
%   embedding of uniform Steiner tree problems --- or for that sake
%   any geometric tree problem --- is automatically handled 
%   this way.} 

% \comment{DW: Although this solves the FST drawing problem, I worry
%   that it may introduce subtle difficulties.  Consider the case of a
%   rectilinear MST edge.  By representing it as a pair of line
%   segments, we have arbitrarily decided whether to make it a
%   ``left-most'' corner or a ``right-most'' corner.  This could impact
%   our ability to prune the edge.  I would rather leave the embedding
%   representation unspecified and force the user to instead provide a
%   structure containing function pointers that we invoke to perform
%   necessary operations such as drawing, various metric-dependent
%   pruning operations, etc.  (Perhaps some of these operations would
%   permit the extraction of a set of line segments that would comprise
%   an embedding.)  It should be valid for the user to use NULL pointers
%   for operations that he does not use.  Specifying a {\em rectilinear}
%   or {\em Euclidean} metric (with $k=2$ dimensions) should be
%   sufficient to obtain default versions of these routines.}

% \comment{MZ: The problem you mention above can easily be solved by the
%   pruning procedure if the metric is known (as I have assumed) by
%   ``melting'' the two lines segments into one ``edge''. In general, I
%   would like to avoid users providing structures and function pointers;
%   it is simply a programming style I don't like. But I am willing to
%   accept that the hypergraph only contains information about embedded
%   ``edges'', i.e., their endpoints and their length. Actually, drawing
%   an edge is not something I consider to be part of the library. Do
%   you agree??}

\subsubsection{Problem solution state}

The problem solution state object represents the ``state'' of some
solution process for a given problem instance (see
Section~\ref{solver_functions}). The object can contain zero or more
feasible (though not necessarily optimal) solutions to the problem.
For a given problem instance, several problem solution state objects
may be created. 
% A problem instance that has one of more problem solution state
% objects, should {\em not} be changed. 
A problem solution state object refers to both a problem
instance being solved and a parameter object (from which all
necessary parameter values are obtained), as illustrated in
Figure~\ref{fig:solutionstate}. 
The problem solution state object has type \code{gst\_solver\_ptr}.

\begin{figure}
\begin{center}
\setlength{\unitlength}{1mm}
\begin{picture}(120,60)(0,0)
%\thicklines
\put(35,40){\framebox(50,20)[l]
{\begin{minipage}{5cm}\begin{center}
Problem Solution State\\ \code{gst\_solver\_ptr}
\end{center}\end{minipage}}}
\put(0,0){\framebox(50,20)[l]
{\begin{minipage}{5cm}\begin{center}
Problem Instance\\ \code{gst\_hg\_ptr}
\end{center}\end{minipage}}}
\put(70,0){\framebox(55,20)[l]
{\begin{minipage}{5cm}\begin{center}
Parameter Set\\ \code{gst\_param\_ptr}
\end{center}\end{minipage}}}
\put(45,40){\vector(-1,-1){19.4}}
\put(75,40){\vector(1,-1){19.4}}
\end{picture}
\end{center}
\caption{Problem solution state references a problem instance and a
parameter set.}
\label{fig:solutionstate}
\end{figure}

\subsubsection{Auxiliary objects}

In addition to the four object classes described above, \geosteiner{}
uses objects for handling metrics, property lists, messages,
and scaling information. A short introduction to these auxiliary 
objects is given in this section.


\paragraph{Metric}
A metric object identifies the method for computing distances between
pairs of points. A metric object has type \code{gst\_metric\_ptr}, and
can be passed as an argument to some of the functions in the callable
library. For more information, see the examples given in
Section~\ref{low_level_interfaces}, and the descriptions of the metric
functions in Section~\ref{metric_functions}.


\paragraph{Property List}
A property list contains auxiliary information about problem
instances and solution state objects, e.g.\ the CPU time for FST
generation (problem instance property) and the current lower bound in
the MSTHG solver (solution state property). 
Property lists have type \code{gst\_proplist\_ptr}, and a property is 
known by its property identification number (see
Section~\ref{proplist_functions}).


\paragraph{Channel}
All output messages from \geosteiner{} are passed through
user-controllable channels. A given channel may write its output to
more than one output (screen/files). Channels have type
\code{gst\_channel\_ptr} and are described in
Section~\ref{message_functions}. 


\paragraph{Scaling Information}
A set of points may have associated scaling information, that is,
information about how the internal representation (double floating
point values) should be scaled back to the original point
coordinates. This is done in order to improve the numerical precision
of \geosteiner{}. Scaling information objects have type
\code{gst\_scale\_info\_ptr} and are described in
Section~\ref{io_functions}. 

%--------------------
\input{functions.tex}
%--------------------

\clearpage
\section{Stand-Alone Programs}
\label{standalone}

%\def\mdescr#1{\begin{minipage}{11cm}{\vspace{2pt}#1\vspace{2pt}}\end{minipage}}
\def\mdescr#1{#1}

Below we first give some examples of program invocations.
This is followed by a complete description of each stand-alone
program.  Note that a short description of each program also can
be obtained by running the program with the {\bf -h} option.

The following command will generate a set of 70 random points and
compute a rectilinear Steiner minimal tree for it:
{\small\begin{verbatim}
 rand_points 70 | rfst | bb
\end{verbatim}
}
\noindent The following computes an Euclidean Steiner minimal tree 
{\small\begin{verbatim}
 rand_points 70 | efst | bb
\end{verbatim}
}
\noindent and the following computes an octilinear Steiner minimal
   tree for the same set of points 
{\small\begin{verbatim}
 rand_points 70 | ufst | bb
\end{verbatim}
}

Note that rand\_points always generates the same sequence of points
unless given the {\bf -r} or {\bf -s} option.

The following (Bourne shell) examples can be used to generate complete
printable postscript plots for these problem instances:
{\small\begin{verbatim}
 (cat prelude.ps; rand_points 70 | rfst | bb) >rsmt70.ps
 (cat prelude.ps; rand_points 70 | efst | bb) >esmt70.ps
 (cat prelude.ps; rand_points 70 | ufst | bb) >usmt70.ps
\end{verbatim}
}
\noindent The complete set of FSTs can also be plotted as follows:
{\small\begin{verbatim}
 (cat prelude.ps; rand_points 70 | rfst | plotfst -fgo) >rfsts.ps
 (cat prelude.ps; rand_points 70 | efst | plotfst -fgo) >efsts.ps
 (cat prelude.ps; rand_points 70 | ufst | plotfst -fgo) >ufsts.ps
\end{verbatim}
}

A reduced Hanan grid in the OR-library format (for the rectilinear
problem) can be generated as follows:
\begin{verbatim}
 rand_points 70 | rfst | fst2graph
\end{verbatim}
\noindent By pruning the set of FSTs, an even more reduced grid graph
can be generated:
\begin{verbatim}
 rand_points 70 | rfst | prunefst | fst2graph
\end{verbatim}

An Euclidean Steiner minimal tree for the \code{berlin52.tsp} instance from
TSPLIB can be constructed and displayed as follows (assuming that the
file \code{berlin52.tsp} is present in your \geosteiner\ directory):
{\small\begin{verbatim}
 (cat prelude.ps; lib_points <berlin52.tsp | efst | bb) | gv -
\end{verbatim}
}

\newpage
\subsection*{rand\_points}

Generates random point sets.
There is considerable flexibility in choosing the size, precision and
scaling factor for the generated point coordinates.
By default, the coordinates are almost always real numbers, uniformly
distributed in the interval $[0,1)$ (see below for exceptions to this
rule).
Several pseudo-random generator algorithms are supported.
The number of digits per coordinate (both default and maximum) vary by
generator, as described below.
The following options are permitted:

\begin{tabular}{lp{11cm}}
\bf -b  & \mdescr{ Binary mode.  Generates coordinates that are
  uniformly distributed doubles in $\lbrack 0,1)$, outputting them
  with full precision. } \\
\bf -d N  & \mdescr{ Generate decimal numbers having N digits.
	(See below for default and maximum values, which vary by
	generator.)} \\
\bf -g G & \mdescr{ Use pseudo-random number generator G.
	(See below.) } \\
\bf -k KEY & \mdescr{ Modify default generator seed with KEY, which can
	be arbitrary text. } \\
\bf -p N & \mdescr{ Make N of the coordinate digits be fractional
  (i.e., to the right of the decimal point).
  Default is for all digits to be fractional.} \\
\bf -r  & \mdescr{Randomize. Use an initial seed chosen from the current date
  and time.} \\
\bf -s FILE & \mdescr{ If FILE exists, read the generator and its
	initial seed from this file.
	When finished, write the generator and final seed to this
        file. }
\end{tabular}

The {\bf -g G} argument allows choosing between the following
pseudo-random number generators:

\begin{tabular}{lcp{10cm}}
	& {\bf Default : Max} & \\
{\bf G}	& {\bf Digits} & {\bf Description} \\ \hline
\bf 0 & 4 : 5 & \mdescr{The ``legacy'' random generator.
	It is based on the original PDP-11 Unix rand(3) function (with
        all of its ugly warts intact).
	Its randomness is quite poor. } \\
\bf 1 & 7 : 9 & \mdescr{The ``new'' random generator.
	It uses a 64-bit shift register with XOR feedback, and
        produces a reasonable level of randomness. } \\
\bf 2 & 7 : 19 & \mdescr{The ``AES-256'' random generator.
	It uses the AES-256 block cipher as its fundamental entropy
        source, producing truly excellent randomness.
	GeoSteiner must be built with GMP in order for this generator
        to be available.} \\ \hline
\end{tabular}
\vskip 0.25in

If no generator is specified, rand\_points will use
generator 2 (AES-256), if available.
Otherwise, generator 1 is used.
The default generator can be overridden using the
RAND\_POINTS\_DEFAULT\_GENERATOR
environment variable.

Note that using -s to load an intial seed from an existing seed file
has the effect of specifying the generator, since there is a data
field within the seed file that specifies the generator.
(This data field is necessary because the format of the seed file
state information is different for each generator.)
If both -g and -s are specified, then either (1) the seed file must
{\bf not} yet exist, or (2) the generator specified with -g must match
that specified within the seed file.

Previous versions of rand\_points only supported generators 0 and 1,
with the default being to use generator 0 to generate 4-digit integer
coordinates.
As a special case, when using generator 0 (legacy) with neither the
-d nor -p arguments, the current version of rand\_points also generates
4-digit integers, thereby replicating the behavior of previous
versions of rand\_points.
Users who prefer the point sets produced by previous versions of
rand\_points can obtain these same, familiar point sets without any
additional command line arguments simply by setting the environment
variable
\begin{displaymath}
\hbox{RAND\_POINTS\_DEFAULT\_GENERATOR=0}
\end{displaymath}

\newpage
\subsection*{lib\_points}

Reads a point set in either TSPLIB or OR-library format from stdin and
converts the input to point coordinates as required by {\bf efst},
{\bf rfst} or {\bf ufst}.  The program automatically
determines the input file type.  The program has one optional
parameter (which has value 1 by default) that specifies which instance
number should be extracted from an OR-library file.

\newpage
\subsection*{efst}

Reads a point set from stdin, and generates a set of Euclidean FSTs
that contains at least one Euclidean Steiner minimal tree. The
following options are permitted:

\bigskip
\begin{tabular}{lp{11cm}}
\bf -d txt      & \mdescr{Description of problem instance.} \\
\bf -g  & \mdescr{Use greedy heuristic instead of Smith-Lee-Liebman (more time
  consuming but generates fewer eq-points).} \\
\bf -k K        & \mdescr{Generate only FSTs having at most K terminals. This can
  save considerable time but can also eliminate FSTs that must be in
  the optimal Steiner tree (i.e., solutions can become suboptimal).} \\
\bf -m M        & \mdescr{Use multiple precision.  Larger M use it more. Default is
  M=0 which disables multiple precision. The use of this option
  requires that \geosteiner\ be configured to use the GNU
  Multi-Precision arithmetic library (GMP).  (See the INSTALL file for
  more details).} \\
\bf -t  & \mdescr{Print detailed timings to stderr.} \\
\bf -v N        & \mdescr{Generate the output in version N of the FST data
  format. Supported versions are 0, 1, 2 and 3. Version 3 is the
  default. } \\
\bf -Z P V & \mdescr{Set parameter P to value V, e.g.\
\mbox{\code{-ZEPS\_MULT\_FACTOR 64}} sets the epsilon
multiplication factor to 64 (\code{GST\_PARAM\_EPS\_MULT\_FACTOR~=~64}).}
\end{tabular}\bigskip

\newpage
\subsection*{rfst}

Reads a point set from stdin, and generates a set of rectilinear FSTs
that contains at least one rectilinear Steiner minimal tree. The
following options are permitted: 

\bigskip
\begin{tabular}{lp{11cm}}
\bf -d txt      & \mdescr{Description of problem instance.} \\
\bf -k K        & \mdescr{Generate only FSTs having at most K terminals. This can
  save time but can also eliminate FSTs that must be in the optimal
  Steiner tree (i.e., solutions can become suboptimal). } \\
\bf -t  & \mdescr{Print detailed timings to stderr.} \\
\bf -v N        & \mdescr{Generate the output in version N of the FST data
  format. Supported versions are 0, 1, 2 and 3. Version 3 is the
  default. } \\
\bf -Z P V & \mdescr{Set parameter P to value V, e.g.\
\mbox{\code{-ZINCLUDE\_CORNERS 0}} disables the generation of corner points
(\code{GST\_PARAM\_INCLUDE\_CORNERS~=~0})}
\end{tabular}\bigskip

\newpage
\subsection*{ufst}

Reads a point set from stdin, and generates a set of uniformly-oriented
FSTs that contains at least one uniformly-oriented Steiner minimal
tree. The following options are permitted: 

\bigskip
\begin{tabular}{lp{11cm}}
\bf -d txt      & \mdescr{Description of problem instance.} \\
\bf -k K        & \mdescr{Generate only FSTs having at most K terminals. This can
  save time but can also eliminate FSTs that must be in the optimal
  Steiner tree (i.e., solutions can become suboptimal).} \\
\bf -l L        & \mdescr{Number of orientations (default: 4).} \\
\bf -t  & \mdescr{Print detailed timings to stderr.} \\
\bf -v N        & \mdescr{Generate the output in version N of the FST data
  format. Supported versions are 0, 1, 2 and 3. Version 3 is the
  default. } \\
\bf -Z P V & \mdescr{Set parameter P to value V, e.g.\
\mbox{\code{-ZINCLUDE\_CORNERS 0}} disables the generation of corner points
(\code{GST\_PARAM\_INCLUDE\_CORNERS~=~0})}
\end{tabular}\bigskip

\newpage
\subsection*{bb}

The FST concatenation algorithm using branch-and-cut to solve an IP formulation
of the problem.  The FST data is read from stdin and a plot of the solution is
produced on stdout in an ``incomplete'' postscript form.  A printable
postscript file can be obtained by {\em prepending} the file "prelude.ps" to
the program output. If you want this file to be included in some other document
then it needs a bounding box. This can be obtained by running it through
{\bf eps2eps} (GhostScript 6.01 or later).

Various trace messages appear in the output as postscript
comments.  (The name {\bf bb} is for branch-and-bound -- note that the name
{\bf bc} is already taken on Unix.)  The following options are permitted:

\bigskip
\begin{tabular}{lp{11cm}}
\bf -2 & \mdescr{Omit all 2-terminal Subtour Elimination Constraints
  (SEC's) from the initial constraint pool.} \\ 
\bf -b & \mdescr{Disable "strong branching", which chooses branching variables
        very carefully.} \\
\bf -B N & \mdescr{Set branch variable selection policy.
         N=0: naive max of mins,
         N=1: smarter lexicographic max of mins (default),
         N=2: product of improvements.} \\
\bf -c P & \mdescr{Pathname of checkpoint file to restore (if present)
         and/or update.  The files are actually named P.chk and P.ub, with
         temporary files named P.tmp, P.new and P.nub.}\\
\bf -f & \mdescr{The only information dumped is the FSTs in the best solution found. This
      can then be given to dumpfst/plotfst. E.g.
 \code{rand\_points | efst | bb -f | dumpfst -sl}}\\
\bf -H & \mdescr{Force the use of the backtrack search. This will result in an
      error if there are more than 32 edges. Note that there is still a limit
      on the number of backtracks (\code{GST\_PARAM\_MAX\_BACKTRACKS}). If
      using this option one might also want to set backtrack limit to infinity
      (otherwise an optimal solution might not be found).}\\
\bf -l T & \mdescr{Sets a CPU time limit (in seconds) of T. Example CPU times are:
  \code{-l 3days2hours30minutes15seconds}, \code{-l 1000seconds},
  \code{-l 1000} and \code{-l 2h30m}.}\\
\bf -m P & \mdescr{Merge constraints from checkpoint file P with those
      of the formulation.}
\end{tabular}


\begin{tabular}{lp{11cm}}
\bf -n N & \mdescr{Output N best solutions (default: 1).}\\
\bf -r   & \mdescr{Plot the optimal LP relaxation solution for the root node,
        but only if it is fractional.} \\
\bf -R   & \mdescr{When optimal root LP relaxation is obtained, determine for
        each LP iteration the number of final constraints whose first violation
        occurred during that iteration.} \\
\bf -t   & \mdescr{Do {\em not} include the title string in the postscript output
                   (name, length and time).} \\
\bf -T N & \mdescr{Search N times more thoroughly for strong branching
        variables.} \\
\bf -u B & \mdescr{Specify B to be the initial upper bound assumed by the
        branch-and-bound.} \\
\bf -z N & \mdescr{Set the target number of pool non-zeros to N.} \\
\bf -Z P V & \mdescr{Set parameter P to value V, e.g.\
\mbox{\code{-ZGAP\_TARGET 0.5}} sets \code{GST\_PARAM\_GAP\_TARGET~=~0.5}.}
\end{tabular}\bigskip

When configured to use CPLEX, the following additional
option is permitted:

\bigskip
\begin{tabular}{lp{11cm}}
\bf -a M N & \mdescr{Force CPLEX allocation to be at least M rows and N
        non-zeros.}
\end{tabular}\bigskip

When configured to use {\code lp\_solve}, the following additional
options are permitted:

\bigskip
\begin{tabular}{lp{11cm}}
\bf -p & \mdescr{Turn on the use of perturbations.  This is the method
        that {\code lp\_solve\_2.3} uses to deal with degenerate problems.} \\
\bf -s & \mdescr{Turn on the use of problem scaling.  Once again a rather
        crude attempt to address problems that are badly behaved numerically.}
\end{tabular}\bigskip

The following ``grep-able'' items appear in the output within
postscript comments, and may be potentially useful:

\bigskip
\begin{itemize}

\item[\code{@0}] The instance description from the FST data file.\\

\item[\code{@1}] Summary statistics:
 \begin{itemize}
  \item Number of terminals
  \item Number of FSTs/hyperedges
  \item Number of branch-and-bound nodes
  \item Number of LPs solved
  \item Phase 1 CPU time (FST generation)
  \item Phase 2 CPU time (branch-and-cut)
  \item Total CPU time
\end{itemize}

\item[\code{@2}] LP/IP statistics:
 \begin{itemize}
  \item Length of optimal Steiner tree
  \item Length of LP relaxation at root node
  \item Percent of LP/IP "gap"
  \item \# of LPs solved for root node
  \item CPU time for root node
  \item Percent reduction of SMT over MST
\end{itemize}

\item[\code{@3}]        Initial constraint pool statistics:
 \begin{itemize}
  \item Number of rows in pool
  \item Number of non-zeros in pool
  \item Number of rows in LP tableau
  \item Number of non-zeros in LP tableau
\end{itemize}

\item[\code{@4}]        Constraint pool statistics for root node:
 \begin{itemize}
  \item Number of rows in pool
  \item Number of non-zeros in pool
  \item Number of rows in LP tableau
  \item Number of non-zeros in LP tableau
\end{itemize}

\item[\code{@5}]        Final constraint pool statistics:
 \begin{itemize}
  \item Number of rows in pool
  \item Number of non-zeros in pool
  \item Number of rows in LP tableau
  \item Number of non-zeros in LP tableau
\end{itemize}

\item[\code{@6}]        Statistics on FSTs occurring in the SMT:
 \begin{itemize}
  \item Number of FSTs in SMT
  \item Average FST size in SMT
  \item Maximum FST size in SMT
  \item Number of FSTs of size 2 in SMT
  \item Number of FSTs of size 3 in SMT
  \item Number of FSTs of size 4 in SMT
  \item Number of FSTs of size 5 in SMT
  \item Number of FSTs of size 6 in SMT
  \item Number of FSTs of size 7 in SMT
  \item Number of FSTs of size 8 in SMT
  \item Number of FSTs of size 9 in SMT
  \item Number of FSTs of size 10 in SMT
  \item Number of FSTs of size $>$ 10 in SMT
\end{itemize}

\item[\code{@C}]           Coordinates of a Steiner point in the optimal
        solution.  The Steiner points form a
        "certificate" of the optimal solution since the
        optimal solution can be reconstructed by
        computing a minimum spanning tree of the
        original terminals and these Steiner points.
\item[\code{@D}]    Deletion of slack rows from LP tableau.
\item[\code{@LO} / \code{@LN}]   This pair of messages is emitted every time the
        lower bound gets tighter.  They contain the CPU
        time and the old/new bound, as well as the
        old/new gap percentages.  These can be plotted
        (i.e., using gnuplot) to graphically show the
        convergence rate of the algorithm.
\item[\code{@NC}]   Creation of a new branch-and-bound node:

\begin{itemize}
  \item Node number
  \item Parent node number
  \item Branch variable
  \item Branch direction
  \item Objective value (the real LP objective is at least this value)
\end{itemize}

\item[\code{@PAP}]  Adding "pending" pool constraints to the LP
        tableau.
\item[\code{@PL}]   State of LP tableau constraints.
\item[\code{@PMEM}] Constraint pool memory status.  Printed before
                   and after each garbage collection, and after
                   adding new/initial constraints to the pool.
\item[\code{@r}]           Experimental output from -R switch.
\item[\code{@RC}]   Experimental output from -R switch.
\item[\code{@UO} / \code{@UN}]  This pair of messages is emitted every time the
        upper bound gets tighter.  They contain the CPU
        time and the old/new bound, as well as the
        old/new gap percentages.  These can be plotted
        (i.e., using gnuplot) to graphically show the
        convergence rate of the algorithm.
\end{itemize}

\newpage
\subsection*{prunefst}

Reduce the set of FSTs generated by {\bf efst}, {\bf rfst} or {\bf
ufst} while still retaining at least one optimal solution among the
remaining set of FSTs.  This program can reduce the time to solve the
FST concatenation problem considerably, but is only worthwhile for
large instances.  The following options are permitted:

\bigskip
\begin{tabular}{lp{11cm}}
\bf -b          & \mdescr{Use linear space and logarithmic time lookup for BSDs.} \\
\bf -d txt      & \mdescr{Description of problem instance.} \\
\bf -t  & \mdescr{Print detailed timings to stderr.} \\
\bf -v N        & \mdescr{Generate the output in version N of the FST data
  format. Supported versions are 0, 1, 2 and 3. Version 3 is the
  default.}\\
\bf -Z P V & \mdescr{Set parameter P to value V, e.g.\
\mbox{\code{-ZEPS\_MULT\_FACTOR 64}} sets the epsilon
multiplication factor to 64 (\code{GST\_PARAM\_EPS\_MULT\_FACTOR~=~64}).}
\end{tabular}\bigskip

\newpage
\subsection*{dumpfst}
Dumps readable information about generated FSTs.  There are two forms
of this command, each producing a different type of output.  The first
form of the command is obtained whenever the {\bf -d} or {\bf -h}
switches are used.  These switches provide summary information {\em only}
--- FST statistics, and/or a histogram of FST sizes:.

\bigskip
\begin{tabular}{lp{11cm}}
\bf -d  & \mdescr{Display statistics about FSTs.} \\
\bf -h  & \mdescr{Display histogram of FST sizes.} \\
\bf -a  & \mdescr{Include all FSTs in histogram, even those that were ``pruned''
  by the FST generator or a pruning algorithm.}
\end{tabular}\bigskip

The second form of the command is obtained when neither {\bf -d} nor
{\bf -h} are specified.  This form dumps all of the FSTs in a readable
form.  Each line of output represents a single FST, listing its
terminal numbers (0 through N-1).  The terminals are listed in the
same order that they occur in the actual data structures, although
they can optionally be sorted in numeric order.  The length of each
FST can optionally be appended to each line:

\bigskip
\begin{tabular}{lp{11cm}}
\bf -l  & \mdescr{Append the FST length to each output line.} \\
\bf -s  & \mdescr{Terminals of each FST are listed in numeric (sorted) order
  instead of internal order.} \\
\bf -a  & \mdescr{Include all FSTs, even those that were ``pruned'' by the FST
  generator or a pruning algorithm.}
\end{tabular}\bigskip

\newpage
\subsection*{plotfst}
Program to generate various plots of FSTs in an FST data file.  Reads
the FST data file on stdin and produces postscript on stdout for the
plots indicated by the command line switches:

\bigskip
\begin{tabular}{lp{11cm}}

\bf -f  & \mdescr{Prints all FSTs, 12 FSTs per page.} \\
\bf -g  & \mdescr{Prints FSTs in ``grouped'' fashion, 12 groups per page.} \\
\bf -o  & \mdescr{Prints all FSTs overlaid together.} \\
\bf -p  & \mdescr{Prints only the points, no FSTs.}
\end{tabular}\bigskip

Note that the file \code{prelude.ps} must be {\em prepended} to the output of
this program in order to have a complete postscript document.

\newpage
\subsection*{fst2graph}

Reads FSTs from stdin and produces an (ordinary) graph on stdout
representing the FSTs.  For the rectilinear problem, the FSTs are
overlaid on the Hanan grid and the remaining Hanan grid is output.
For the Euclidean problem the set of all terminals and Steiner points
in all FSTs forms the set of vertices and the line segments form the
edges. Output data is printed in the OR-library format by default, but the
SteinLib format is also supported: 

\bigskip
\begin{tabular}{lp{11cm}}
\bf -b N	& \mdescr{For version 4 output (STEINLIB\_INT),
 generate integer edge weights as unsigned N-bit integers
 (default is N=64).} \\
\bf -d txt      & \mdescr{Description of problem instance.} \\
\bf -e  & \mdescr{Generate the edge graph for the rectilinear problem.} \\
\bf -u  & \mdescr{Output unscaled (fractional) data for the rectilinear
  problem.} \\
\bf -v N	& \mdescr{Generate version N format output.}
\end{tabular}\bigskip

\newpage
\addcontentsline{toc}{section}{References}
\bibliographystyle{plain}
% \bibliography{/net/verdande/home/disk00/martinz/project/text/mz_references}

\begin{thebibliography}{10}

\bibitem{KahngRobins92a}
A.~B. Kahng and G.~Robins.
\newblock {A New Class of Iterative Steiner Tree Heuristics with Good
  Performance}.
\newblock {\em IEEE Transactions on Computer-Aided Design of Integrated
  Circuits and Systems}, 11(7):893--902, 1992.

\bibitem{NielsenWinterZachariasen02c}
B.~K. Nielsen, P.~Winter, and M.~Zachariasen.
\newblock {An Exact Algorithm for the Uniformly-Oriented Steiner Tree Problem}.
\newblock In {\em Proceedings of the 10th European Symposium on Algorithms,
  Lecture Notes in Computer Science}, volume 2461, pages 760--772. Springer,
  2002.

\bibitem{SaloweWarme95}
J.~S. Salowe and D.~M. Warme.
\newblock {Thirty-Five-Point Rectilinear Steiner Minimal Trees in a Day}.
\newblock {\em Networks}, 25(2):69--87, 1995.

\bibitem{SmithLeeLiebman81}
J.~M. Smith, D.~T. Lee, and J.~S. Liebman.
\newblock {An $O(n \log n)$ Heuristic for Steiner Minimal Tree Problems on the
  Euclidean Metric}.
\newblock {\em Networks}, 11:23--29, 1981.

\bibitem{Warme98}
D.~M. Warme.
\newblock {\em {Spanning Trees in Hypergraphs with Applications to Steiner
  Trees}}.
\newblock Ph.D. thesis, Computer Science Dept., The University of Virginia, 1998.

\bibitem{WarmeWinterZachariasen00}
D.~M. Warme, P.~Winter, and M.~Zachariasen.
\newblock {Exact Algorithms for Plane Steiner Tree Problems: A Computational
  Study}.
\newblock In D.-Z. Du, J.~M. Smith, and J.~H. Rubinstein, editors, {\em
  {Advances in Steiner Trees}}, pages 81--116. Kluwer Academic Publishers,
  Boston, 2000.

\bibitem{Winter85}
P.~Winter.
\newblock {An Algorithm for the Steiner Problem in the Euclidean Plane}.
\newblock {\em Networks}, 15:323--345, 1985.

\bibitem{WinterZachariasen97}
P.~Winter and M.~Zachariasen.
\newblock {Euclidean Steiner Minimum Trees: An Improved Exact Algorithm}.
\newblock {\em Networks}, 30:149--166, 1997.

\bibitem{Zachariasen99a}
M.~Zachariasen.
\newblock {Rectilinear Full Steiner Tree Generation}.
\newblock {\em Networks}, 33:125--143, 1999.

\bibitem{ZachariasenWinter98}
M.~Zachariasen and P.~Winter.
\newblock {Concatenation-Based Greedy Heuristics for the Euclidean Steiner Tree
  Problem}.
\newblock {\em Algorithmica}, 25:418--437, 1999.

\end{thebibliography}

\clearpage
\appendix

\def\param#1{\noindent{\bf\ttfamily GST\_PARAM\_#1}}
\def\pname#1{\noindent\hrulefill\\ \noindent{\bf\ttfamily GST\_PARAM\_#1}}
\def\ptype#1{\vspace*{-0.9cm}\begin{flushright}\code{#1}\end{flushright}}
\def\pdescr#1{\noindent #1 \bigskip}
\def\pvalhead{\noindent{\bf Values}\\}
\def\pval#1#2#3{\noindent\makebox[10.5cm][l]{\code{GST\_PVAL\_#1}}\makebox[1cm][r]{#2}\makebox[2cm][r]{#3}}

\section{Library Parameters}
\label{parameters}

The parameters in \geosteiner\ are divided into five groups: FST
generation parameters (Section~\ref{parameters_fst}), LP solver
parameters (Section~\ref{parameters_lp}), hypergraph solver
algorithmic options (Section~\ref{parameters_solver_alg}), 
hypergraph solver stopping conditions (Section~\ref{parameters_stop}),
and  hypergraph solver input/output options
(Section~\ref{parameters_io}). 

Parameters are modified as described in
Section~\ref{parameter_functions}.  
Each parameter is uniquely identified by its macro name beginning with
(\code{GST\_PARAM}). Below the effect of each parameter is
described. Also, the type of each parameter (\code{double}, \code{int},
\code{char*} or \code{gst\_channel\_ptr}) and range of possible
values are given.

% ----------------------------------------------------------------------
% \subsection{Hypergraph parameters} 
% \label{parameters_hg}

% \pname{GROUP\_DEFINITION}
% \ptype{int}

% \pdescr{Definition of vertex groups: Either at least one or exactly one vertex
% should be chosen for each group.}

% \pvalhead
% \pval{GROUP\_DEFINITION\_ATLEAST}{0}{(default)}\\
% \pval{GROUP\_DEFINITION\_EXACTLY}{1}{}

% ----------------------------------------------------------------------
\newpage
\subsection{FST generation parameters}
\label{parameters_fst}


% ----------------------------------------------------------------------
\pname{MAX\_FST\_SIZE}
\ptype{int} 

\pdescr{Maximum size (number of terminals spanned) of generated FSTs.}

\pvalhead
Any number greater than or equal to 2 (default: \code{INT\_MAX}).

% ----------------------------------------------------------------------
\pname{INCLUDE\_CORNERS}
\ptype{int}

\pdescr{Include corners of bent edges in FSTs in hypergraph
  embedding. Applies to rectilinear and uniform-orientation metric FST 
  generators. Including corners makes the embedding easier to
  draw.}

\pvalhead
\pval{INCLUDE\_CORNERS\_DISABLE}{0}{(default)}\\
\pval{INCLUDE\_CORNERS\_ENABLE}{1}{}

% ----------------------------------------------------------------------
\pname{EFST\_HEURISTIC}
\ptype{int}

\pdescr{Heuristic used in the Euclidean FST generator:
  Smith-Lee-Liebman or Zachariasen-Winter. The latter is slower but
  prunes more eq-points; it is therefore recommended for large and/or
  difficult instances.} 

\pvalhead
\pval{EFST\_HEURISTIC\_SLL}{0}{(default)}\\
\pval{EFST\_HEURISTIC\_ZW}{1}{}

% ----------------------------------------------------------------------
\pname{EPS\_MULT\_FACTOR}
\ptype{int}

\pdescr{Epsilon multiplication factor F used in floating point
  comparisons. The maximum relative error is expected to be at most 
  F $\times$ \code{DBL\_ESPILON}.}

\pvalhead
Any number greater than or equal to 1 (default: 32).

% ----------------------------------------------------------------------
\pname{MULTIPLE\_PRECISION}
\ptype{int}

\pdescr{Use GNU Multi-Precision arithmetic library (GMP) in the
  Euclidean FST generator in order to improve numerical
  precision of computed eq-points: 0:~off; 1:~use GMP
  with 1 Newton iteration; 2:~use GMP with 1 or more Newton
  iterations, stopping when a convergence test indicates that 1/2 ULP
  of precision has been obtained.}

\pvalhead
\pval{MULTIPLE\_PRECISION\_OFF}{0}{(default)}\\
\pval{MULTIPLE\_PRECISION\_ONE\_ITER}{1}{}\\
\pval{MULTIPLE\_PRECISION\_MORE\_ITER}{2}{}

% ----------------------------------------------------------------------
\pname{INITIAL\_EQPOINTS\_TERMINAL}
\ptype{int}

\pdescr{Number of eq-points initially allocated per terminal in
  the Euclidean FST generator.  Although eq-point storage is added
  dynamically when needed, some large or difficult instances run out
  of memory if the initial allocation is insufficient.}

\pvalhead
Any number greater than or equal to 1 (default: 100).

% ----------------------------------------------------------------------
\pname{BSD\_METHOD}
\ptype{int}

\pdescr{Data structure for holding bottleneck Steiner distances
(BSD). Either quadratic space and \emph{constant} time lookup or linear
space and \emph{logarithmic} time lookup. The latter is recommended for
very large instances.}

\pvalhead
\pval{BSD\_METHOD\_CONSTANT}{0}{(default)}\\
\pval{BSD\_METHOD\_LOGARITHMIC}{1}{}

% ----------------------------------------------------------------------
\newpage
\subsection{LP solver parameters} 
\label{parameters_lp}

% ----------------------------------------------------------------------
\pname{LP\_SOLVE\_PERTURB}
\ptype{int}

\pdescr{Use perturbations when solving LPs (only applicable
  when using lp\_solve as LP-solver).} 

\pvalhead
\pval{LP\_SOLVE\_PERTURB\_DISABLE}{0}{(default)}\\
\pval{LP\_SOLVE\_PERTURB\_ENABLE}{1}{}

% ----------------------------------------------------------------------
\pname{LP\_SOLVE\_SCALE}
\ptype{int}

\pdescr{Use scaling when solving LPs (only applicable when using
  lp\_solve as LP-solver).}

\pvalhead
\pval{LP\_SOLVE\_SCALE\_DISABLE}{0}{(default)}\\
\pval{LP\_SOLVE\_SCALE\_ENABLE}{1}{}

% ----------------------------------------------------------------------
\pname{CPLEX\_MIN\_ROWS}
\ptype{int}

\pdescr{Force the LP solver allocation to be at least N rows (only
  applicable when using CPLEX as LP-solver).}

\pvalhead
Any non-negative number (default: 0).

% ----------------------------------------------------------------------
\pname{CPLEX\_MIN\_NZS}
\ptype{int}

\pdescr{Force the LP solver allocation to be at least N non-zeros (only
  applicable when using CPLEX as LP-solver).} 

\pvalhead
Any non-negative number (default: 0).

% ----------------------------------------------------------------------
\newpage
\subsection{Hypergraph solver algorithmic options}
\label{parameters_solver_alg}

% ----------------------------------------------------------------------
\pname{SOLVER\_ALGORITHM}
\ptype{int}

\pdescr{Hypergraph solver algorithm: Branch-and-cut, backtrack search,
  or chosen automatically. Backtrack search is only applicable if the
  instance has 32 or fewer hyperedges. Also note that some stopping
  conditions --- such as UB/LB gap --- are {\em not} feasible for
  backtrack search. 
  The automatic algorithm uses backtrack search when the instance is
  small (see parameters
  \code{GST\_PARAM\_BACKTRACK\_MAX\_VERTS} and
  \code{GST\_PARAM\_BACKTRACK\_MAX\_EDGES}); furthermore, it switches 
  to branch-and-cut when the the backtrack limit
  \code{GST\_PARAM\_MAX\_BACKTRACKS} is hit. }

\pvalhead
\pval{SOLVER\_ALGORITHM\_AUTO}{0}{(default)}\\
\pval{SOLVER\_ALGORITHM\_BRANCH\_AND\_CUT}{1}{}\\
\pval{SOLVER\_ALGORITHM\_BACKTRACK\_SEARCH}{2}{}

% ----------------------------------------------------------------------
\pname{NUM\_FEASIBLE\_SOLUTIONS}
\ptype{int}

\pdescr{Number N of stored feasible solutions (top N solutions).  A
  value of N for this parameter instructs the solver to retain the N
  best feasible solutions discovered.}

\pvalhead
Any number greater than or equal to 1 (default: 1).

% ----------------------------------------------------------------------
\pname{BRANCH\_VAR\_POLICY}
\ptype{int}

\pdescr{Branch variable policy. 0:~naive max of mins, 1:~smarter
  lexicographic max of mins, 2:~product of improvements; 3:~weak
  branching. All policies except the last one use strong branching.}

\pvalhead
\pval{BRANCH\_VAR\_POLICY\_NAIVE}{0}{}\\
\pval{BRANCH\_VAR\_POLICY\_SMART}{1}{(default)}\\
\pval{BRANCH\_VAR\_POLICY\_PROD}{2}{}\\
\pval{BRANCH\_VAR\_POLICY\_WEAK}{3}{}

% ----------------------------------------------------------------------
\pname{CHECK\_BRANCH\_VARS\_THOROUGHLY}
\ptype{int} 

\pdescr{Search N times more thoroughly for strong branching
  variables.}

\pvalhead
Any number from 1 to 1000 (default: 1).

% ----------------------------------------------------------------------
\pname{TARGET\_POOL\_NON\_ZEROS}
\ptype{int} 

\pdescr{Target number of pool non-zeros; target is computed
  automatically when value is zero.}

\pvalhead
Any non-negative number (default: 0).

% ----------------------------------------------------------------------
\pname{SEED\_POOL\_WITH\_2SECS}
\ptype{int}

\pdescr{This parameter controls whether or not to seed the initial
constraint pool with all 2-terminal Subtour Elimination Constraints
(SECs). Most problems have relatively few of these, but some problems
(such as those with many edges containing a large number of vertices)
can blow up unless this 
is disabled.}

\pvalhead
\pval{SEED\_POOL\_WITH\_2SECS\_DISABLE}{0}{}\\
\pval{SEED\_POOL\_WITH\_2SECS\_ENSABLE}{1}{(default)}

% ----------------------------------------------------------------------
\pname{INITIAL\_UPPER\_BOUND}
\ptype{double}

\pdescr{Value of initial upper bound for problem being solved.}

\pvalhead
Any number (default: \code{DBL\_MAX}).

% ----------------------------------------------------------------------
\pname{CHECK\_ROOT\_CONSTRAINTS}
\ptype{int}

\pdescr{When the optimal root LP relaxation is obtained, determine for
  each LP iteration the number of final constraints whose first
  violation occurred during that iteration.
  This option creates a temporary file to hold the LP solution vector
  from each iteration.  This file can grow very large.}

\pvalhead
\pval{CHECK\_ROOT\_CONSTRAINTS\_DISABLE}{0}{(default)}\\
\pval{CHECK\_ROOT\_CONSTRAINTS\_ENABLE}{1}{}

% ----------------------------------------------------------------------
\pname{LOCAL\_CUTS\_MODE}
\ptype{int} 

\pdescr{Local cuts mode: 
0:~disable local cuts; 
1:~apply local cuts only when no subtour violation exists; 
2:~apply local cuts to congested components that contain no subtour
violations; 
3:~apply local cuts in both cases.} 

\pvalhead
\pval{LOCAL\_CUTS\_MODE\_DISABLE}{0}{(default)}\\
\pval{LOCAL\_CUTS\_MODE\_SUBTOUR\_RELAXATION}{1}{}\\
\pval{LOCAL\_CUTS\_MODE\_SUBTOUR\_COMPONENTS}{2}{}\\
\pval{LOCAL\_CUTS\_MODE\_BOTH}{3}{}

% ----------------------------------------------------------------------
\pname{LOCAL\_CUTS\_MAX\_VERTICES}
\ptype{int} 

\pdescr{Local cuts will not be attempted for any subproblem having
  more than this number of vertices.}

\pvalhead
Any number from 0 to 80 (default: 80).

% ----------------------------------------------------------------------
\pname{LOCAL\_CUTS\_MAX\_EDGES} 
\ptype{int}

\pdescr{Local cuts will not be attempted for any subproblem having
  more than this number of edges.}

\pvalhead
Any number from 0 to 256 (default: 256).

% ----------------------------------------------------------------------
\pname{LOCAL\_CUTS\_VERTEX\_THRESHOLD}
\ptype{double} 

\pdescr{A threshold value $\alpha$ that prohibits local cuts on any
  fractional component $C = (V',E')$ of a parent problem $H = (V,E)$
  unless $|V'| < \alpha * |V|$.}

\pvalhead
Any number from 0 to 1 (default: 0.75).

% ----------------------------------------------------------------------
\pname{LOCAL\_CUTS\_MAX\_DEPTH} 
\ptype{int}

\pdescr{Maximum recursive depth of local cuts. 
0: disable local cuts;
1: enable local cuts with no recursion;
2: enable local cuts with two recursive levels.
-1: enable local cuts with recursion to any depth.}

\pvalhead
\pval{LOCAL\_CUTS\_MAX\_DEPTH\_DISABLE}{0}{}\\
\pval{LOCAL\_CUTS\_MAX\_DEPTH\_ONELEVEL}{1}{(default)}\\
\pval{LOCAL\_CUTS\_MAX\_DEPTH\_TWOLEVELS}{2}{}\\
\pval{LOCAL\_CUTS\_MAX\_DEPTH\_ANYLEVEL}{-1}{}

% ----------------------------------------------------------------------
\pname{LOCAL\_CUTS\_TRACE\_DEPTH}
\ptype{int}

\pdescr{Tracing of local cuts. 
0:~do not trace local cuts or their recursive invocations;
1:~trace first level of local cuts;
2:~trace first two levels of local cuts;
-1:~trace any level of local cuts.}

\pvalhead
\pval{LOCAL\_CUTS\_TRACE\_DEPTH\_DISABLE}{0}{(default)}\\
\pval{LOCAL\_CUTS\_TRACE\_DEPTH\_ONELEVEL}{1}{}\\
\pval{LOCAL\_CUTS\_TRACE\_DEPTH\_TWOLEVELS}{2}{}\\
\pval{LOCAL\_CUTS\_TRACE\_DEPTH\_ANYLEVEL}{-1}{}

% ----------------------------------------------------------------------
\pname{LOCAL\_CUTS\_TRACE\_STYLE}
\ptype{int}

\pdescr{Selects the style of local cuts trace messages.}

\pvalhead
\pval{LOCAL\_CUTS\_TRACE\_STYLE\_NORMAL}{0}{(default)}\\
\pval{LOCAL\_CUTS\_TRACE\_STYLE\_VERBOSE}{1}{}

% ----------------------------------------------------------------------
\pname{MAX\_CUTSET\_ENUMERATE\_COMPS}
\ptype{int}

\pdescr{Controls the behavior of the zero-weight cutset separation
algorithm, which looks for multiple connected components.  If the
number $N$ of connected components does not exceed this threshold, then
the separator generates one cutset constraint for each of the $2^N-2$
possible combinations of connected components (excluding the two
combinations that take all or none of the components).  This parameter
controls an exponential process, so setting it too high can easily
swamp the solver with constraints.}

\pvalhead
Any number from 0 to 11 (default: 5)

% ----------------------------------------------------------------------
\pname{SEC\_ENUM\_LIMIT}
\ptype{int}

\pdescr{Congested components having at most this number of vertices
are exhaustively searched to find all violated subtour elimination
constraints.  A component with $N$ vertices has $2^N - N - 1$
possible subtour elimination constraints.  This parameter therefore
controls an exponential process --- setting it too high can easily
swamp the solver with constraints or increase runtime.}

\pvalhead
Any number from 0 to 16 (default: 10)

% ----------------------------------------------------------------------
\pname{BACKTRACK\_MAX\_VERTS}
\ptype{int}

\pdescr{Backtrack search should only be attempted for solving MST in 
hypergraph problem when the number of vertices is smaller than this
value.} 

\pvalhead
Any number from 0 to 32 (default: 8).

% ----------------------------------------------------------------------
\pname{BACKTRACK\_MAX\_EDGES}
\ptype{int}

\pdescr{Backtrack search should only be attempted for solving MST in 
hypergraph problem when the number of edges is smaller than this
value.} 

\pvalhead
Any number from 0 to 32 (default: 12).


% ----------------------------------------------------------------------
\pname{MAX\_BACKTRACKS}
\ptype{int}

\pdescr{Maximum number of distinct partial solution nodes to enumerate
during a single run of the backtrack search algorithm. Note that if
this limit is hit, the solver might exit without having found an
optimal solution.} 

\pvalhead
Any non-negative number (default: 10000).


% ----------------------------------------------------------------------
\pname{SPARSE\_SUBTOURS}
\ptype{int}

\pdescr{There are two forms for hypergraph subtour constraints: the
``classic'' subtour, and the form obtained by subtracting the
``classic'' inequality from the single equation of the formulation.
If enabled, then generate the form that has the least number of
non-zeros (i.e., more sparse).  If disabled, always generate the
``classic'' subtour inequality.}

\pvalhead
\pval{SPARSE\_SUBTOURS\_DISABLE}{0}{}\\
\pval{SPARSE\_SUBTOURS\_ENABLE}{1}{(default)}


% ----------------------------------------------------------------------
\pname{ZERO\_WEIGHT\_CUTSETS\_METHOD}
\ptype{int}

\pdescr{Controls the behavior of the zero-weight cutset separation
algorithm, which is invoked when the support hypergraph of the current
LP solution consists of two or more connected components.
For each such connected component, this algorithm generates either:
(1) a pair of complementary subtour inequalities; or
(2) an actual cutset inequality ({\em not} recommended).}

\pvalhead
\pval{ZERO\_WEIGHT\_CUTSETS\_METHOD\_SUBTOURS}{0}{(default)}\\
\pval{ZERO\_WEIGHT\_CUTSETS\_METHOD\_CUTSET}{1}{}


% ----------------------------------------------------------------------
\pname{STRENGTHEN\_REDUCE}
\ptype{int}

\pdescr{If enabled, attempt to strengthen generated subtour inequalities
by performing the various reductions (greedy vertex deletion,
connected components and biconnected components) on the subhypergraph
induced by the vertices of the subtour.}

\pvalhead
\pval{STRENGTHEN\_REDUCE\_DISABLE}{0}{}\\
\pval{STRENGTHEN\_REDUCE\_ENABLE}{1}{(default)}


% ----------------------------------------------------------------------
\pname{GENERATE\_UNSTRENGTHENED}
\ptype{int}

\pdescr{If enabled, generate violated subtours in their original form
--- before any strengthening is performed.
(This is implcitly enabled if \code{GST\_PARAM\_STRENGTHEN\_REDUCE} is
disabled to prevent all violated subtours from being discarded.)}

\pvalhead
\pval{GENERATE\_UNSTRENGTHENED\_DISABLE}{0}{}\\
\pval{GENERATE\_UNSTRENGTHENED\_ENABLE}{1}{(default)}


% ----------------------------------------------------------------------
\pname{INITIAL\_PRIMAL\_HEURISTIC}
\ptype{int}

\pdescr{If enabled, run the primal heuristic at the very start, (before
any LPs have been solved).  Normally the primal heuristic is given the
current LP solution as a ``hint'' for constructing good solutions, but
it can also run no such solution.  This option is normally disabled
because any such upper bound is usually replaced quickly on the next
invocation of the primal heuristic, after the first LP has been solved.
Turning this option on permits a very quick solution.}

\pvalhead
\pval{INITIAL\_PRIMAL\_HEURISTIC\_DISABLE}{0}{(default)}\\
\pval{INITIAL\_PRIMAL\_HEURISTIC\_ENABLE}{1}{}

% ----------------------------------------------------------------------
\newpage
\subsection{Hypergraph solver stopping conditions} 
\label{parameters_stop}

% ----------------------------------------------------------------------
\pname{CPU\_TIME\_LIMIT}
\ptype{double}

\pdescr{CPU time limit for solver (in seconds); when the limit is
  zero, {\em no} CPU time limit is imposed.} 

\pvalhead
Any non-negative number (default: 0).

% ----------------------------------------------------------------------
\pname{GAP\_TARGET}
\ptype{double}

\pdescr{Exit solver when ratio UB/LB between the upper bound (UB) and
  the lower bound (LB) is less than or equal to this threshold; e.g., if
  target is 1.01, the solver stops when a solution within 1\% from the
  optimum has been found.}

\pvalhead
Any number greater than or equal to 1 (default: 1).

% ----------------------------------------------------------------------
\pname{UPPER\_BOUND\_TARGET}
\ptype{double}

\pdescr{Exit solver when a feasible solution whose length is at most
  this value is found.}

\pvalhead
Any number (default: \code{-DBL\_MAX}).

% ----------------------------------------------------------------------
\pname{LOWER\_BOUND\_TARGET} 
\ptype{double} 

\pdescr{Exit solver when the lower bound becomes greater than or equal
  to this value.}

\pvalhead
Any number (default: \code{DBL\_MAX}).

% ----------------------------------------------------------------------
\pname{MAX\_FEASIBLE\_UPDATES}
\ptype{int} 

\pdescr{Exit solver when N feasible solution updates have been made
  (zero means no limit). A feasible update is either an insertion of a
  solution of any quality into the (non-full) set of solutions, or a
  replacement of an inferior solution with an improved solution in the
  (full) set of solutions. The size of the solution set is specified
  using parameter \code{GST\_PARAM\_NUM\_FEASIBLE\_SOLUTIONS}.}

\pvalhead
Any non-negative number (default: 0).


% ----------------------------------------------------------------------
\pname{BB\_NODE\_LIMIT}
\ptype{int}

\pdescr{Stop the optimization after processing this many branch-and-bound
nodes.
If this parameter is zero, there is no limit on the number of
branch-and-bound nodes processed.}

\pvalhead
Any non-negative number (default: 0).


% ----------------------------------------------------------------------
\pname{BB\_LP\_LIMIT}
\ptype{int}

\pdescr{Stop the optimization after processing this many LPs (optimize
/ separate iterations).
The number of LPs processed is a global number that completely
disregards branch-and-bound node boundaries.
If this parameter is zero, there is no limit on the number of LPs
processed.}

\pvalhead
Any non-negative number (default: 0).


% ----------------------------------------------------------------------
\pname{INITIAL\_PRIMAL\_HEUR\_STOP}
\ptype{int}

\pdescr{If this parameter and \code{INITIAL\_PRIMAL\_HEURISTIC} are both
enabled, optimization stops immediately after this initial invocation
of the primal heuristic, with a status of
\code{GST\_SOLVE\_BB\_STOP\_REQUESTED}.
No LPs are solved, nor is any branch-and-bound performed.  This permits
a very quick solution, but no lower bound will be available, and
therefore no indication of solution quality.}

\pvalhead
\pval{INITIAL\_PRIMAL\_HEUR\_STOP\_DISABLE}{0}{(default)}\\
\pval{INITIAL\_PRIMAL\_HEUR\_STOP\_ENABLE}{1}{}


% ----------------------------------------------------------------------
\newpage
\subsection{Hypergraph solver input/output options}
\label{parameters_io}

% ----------------------------------------------------------------------
\pname{SAVE\_FORMAT} 
\ptype{int}

\pdescr{Format used by {\bf gst\_hg\_save()} when saving a hypergraph to a
 file:
0: OR-library format;
1: SteinLib format;
2: GeoSteiner FST format version 2;
3: GeoSteiner FST format version 3;
4: SteinLib format with integer edge weights.}

\pvalhead
\pval{SAVE\_FORMAT\_ORLIBRARY}{0}{}\\
\pval{SAVE\_FORMAT\_STEINLIB}{1}{}\\
\pval{SAVE\_FORMAT\_VERSION2}{2}{}\\
\pval{SAVE\_FORMAT\_VERSION3}{3}{(default)}\\
\pval{SAVE\_FORMAT\_STEINLIB\_INT}{4}{}

% ----------------------------------------------------------------------
\pname{SAVE\_INT\_NUMBITS} 
\ptype{int}

\pdescr{Number of bits of precision to use in final integer edge
weights when using {\bf gst\_hg\_save()} to save hypergraphs having
Euclidean metric problems in ``integer'' SteinLib format
({\bf GST\_PARAM\_SAVE\_FORMAT} set to \\
\noindent
 {\bf GST\_PVAL\_SAVE\_FORMAT\_STEINLIB\_INT}).}

\pvalhead
Value must be at least 32.  Default value is 64.

% ----------------------------------------------------------------------
\pname{GRID\_OVERLAY} 
\ptype{int}

\pdescr{Used by function {\bf gst\_hg\_to\_graph()} to specify that
  the edges of the reduced grid graph rather than individual edges of
  the embedding should be returned (only applicable for the
  rectilinear metric).}

\pvalhead
\pval{GRID\_OVERLAY\_DISABLE}{0}{}\\
\pval{GRID\_OVERLAY\_ENABLE}{1}{(default)}

% ----------------------------------------------------------------------
\pname{DETAILED\_TIMINGS\_CHANNEL} 
\ptype{gst\_channel\_ptr} 

\pdescr{Detailed timing is written to this channel.} 

\pvalhead
Any valid channel pointer (default: \code{NULL}).

% ----------------------------------------------------------------------
\pname{PRINT\_SOLVE\_TRACE} 
\ptype{gst\_channel\_ptr} 

\pdescr{Solver output trace is written to this channel.}

\pvalhead
Any valid channel pointer (default: \code{NULL}).

% ----------------------------------------------------------------------
\pname{CHECKPOINT\_FILENAME}
\ptype{char*} 
\pdescr{Pathname P of checkpoint file to restore (if present) and/or
  update. The files are actually named P.chk and P.ub, with temporary
  files named P.tmp, P.new and P.nub.} 

\pvalhead
Any valid pathname (default: \code{NULL}).

% ----------------------------------------------------------------------
\pname{CHECKPOINT\_INTERVAL}
\ptype{double}

\pdescr{Perform checkpointing of solver process at a time interval (in 
  seconds) given by this parameter.} 

\pvalhead
Any number between 0 and 1000000 (default: 3600). A value of 0 means
that no checkpointing should be performed.

% ----------------------------------------------------------------------
\pname{MERGE\_CONSTRAINT\_FILES}
\ptype{char*}

\pdescr{A colon-separated list of pathnames of checkpoint files.  All
constraints from the constraint pool of each listed checkpoint file
are merged into the solver's constraint pool before solving the
current hypergraph problem.}

\pvalhead
A colon-separated list of pathnames of checkpoint files (default: \code{NULL}).

\clearpage
\section{Hypergraph Properties}
\label{hypergraph_properties}

The following table shows the properties currently accessible in a hypergraph
instance. Read more about properties in Section~\ref{proplist_functions}.

\begin{center}
\input{hgprops.tex}
\end{center}

\clearpage
\section{Solver Properties}
\label{solver_properties}

The following table shows the properties currently accessible in a solver
object. Read more about properties in Section~\ref{proplist_functions}.

\begin{center}
\input{solverprops.tex}
\end{center}

\clearpage
\section{Error Codes}

\input{errors.tex}


\section{FST Data File Formats}
\label{sec:fst_formats}

The FST generators produce output called FST data files.  (They are
sometimes called ``phase 1 data files'', since FST generation is the first
phase of the two-phase process for computing Steiner trees.

FST data files come in three different formats, distinguished by version
numbers.  Currently there are three such formats corresponding to
versions 0, 2 and 3 of the FST data format.  (Version 1 is very
obsolete, and no longer supported.)

Note that version 0 and 3 data formats can be used to describe Steiner
tree in graph (or hypergraph) instances.  However,
\geosteiner~\GeoSteinerCurrentVersion{}
{\em cannot} solve such problems. It blindly assumes all vertices are
terminals.  If given such an instance, \geosteiner\ will produce the MST
(i.e., the minimum tree spanning {\em all} vertices, be they terminals,
Steiner vertices, or any mixture thereof.)

\subsection*{Version 0}

Version 0 is used to represent an abstract MST or Steiner tree in graph
or hypergraph problem instance.  It is essentially the same format as
used in Beasley's ``OR-library'' -- but extended slightly to handle
hypergraph instances as well as graph instances.  The OR-library format
is as follows:

{\footnotesize
\begin{verbatim}
<Number of vertices N> <Number of edges M>
For each edge:
    <Vertex 1> <Vertex 2> <Edge cost>
<Number of terminal vertices K>
    <Terminal 1> <Terminal 2> ... <Terminal K>
\end{verbatim}
}

Vertices are numbered 1 through N.  Each \code{<Terminal i>} is the vertex
number of a vertex that is a terminal (i.e., must be connected).
The \code{<Edge cost>}'s are real numbers.

We extend this format slightly by permitting each edge to have two
{\em or more} vertices.  In exchange for this flexibility, we require the entire
description of each edge to reside on a single line of the data file.
Therefore, the final number on each line represents the hyperedge cost,
and all preceding numbers on the line represent the vertices of the
hyperedge.


\newpage
\subsection*{Version 2}

Version 2 is used primarily to represent geometric FSTs (Euclidean or
rectilinear), although it can also handle non-geometric (graph)
instances.  It is unable, however, to represent Steiner trees in
hypergraph problems, because it assumes every vertex is a terminal.

In the following description, fields enclosed in \code{<<...>>} are
omitted when the Metric is Graph.  The format is as follows:

{\footnotesize
\begin{verbatim}
<Version Number: literally "V2">
<Instance description (free text)>
<Metric: 1 = Rectilinear, 2 = Euclidean, 3 = Graph>
<Number of terminals (N)>
<<Decimal length of MST>> <<Hex length of MST>>
<<Number of duplicate terminal groups (ndg)>>
<Coordinate/length scaling factor>
<Machine description (free text)>
<Front-end CPU-time (1/100s of a second (integer number)>
<Number of hyperedges/FSTs (M)>
For each terminal:
    <<Dec X-coord>> <<Dec Y-coord>> <<Hex X-coord>> <<Hex Y-coord>>
For each duplicate terminal group:
    <<Number of duplicate terminals>>
    <<Terminal indices (1..N), on one line separated by spaces>>
For each hyperedge/FST:
    <Number of terminals (Ni)>
    <Terminal indices (1..N), on one line separated by spaces>
    <Decimal length of hyperedge/FST> <Hex length of hyperedge/FST>
    <<Number of Steiner points (Mi)>>
    For each Steiner point:
       <<Dec X-coord>> <<Dec Y-coord>> <<Hex X-coord>> <<Hex Y-coord>>
    <<Number of FST edges (Ki)>>
    For each FST edge:
       <<endpoint-1>> <<endpoint-2>>
    <FST status: 0 = never needed, 1 = maybe needed, 2 = always needed>
    <Number of incompatible FSTs>
    <Incompatible FST indices (1..M), on one line separated by spaces>
    <Number of concatenation terminals>
    <Conc. terminals indices (1..N), on one line separated by spaces>
\end{verbatim}
}

\newpage
\subsection*{Version 3}

Version 3 is the default format, and represents geometric FSTs
(Euclidean or rectilinear) as well as graph instances.  Since it
separately specifies each vertex to be either a terminal or Steiner
vertex, it can also represent Steiner problems in graphs/hypergraphs.
A number of obsolete fields from version 2 is omitted, however.

In the following description, fields enclosed in \code{<<...>>} are
omitted when the Metric is Graph.  The format is as follows:

{\footnotesize
\begin{verbatim}
<Version Number: literally "V3">
<Instance description (free text)>
<Metric: 1 = Rectilinear, 2 = Euclidean, 3 = Graph>
<Number of terminals (N)>
<<Decimal length of MST>> <<Hex length of MST>>
<Coordinate/length scaling factor>
<Decimal Integrality delta> <Hex Integrality delta>
<Machine description (free text)>
<Front-end CPU-time (1/100s of a second (integer number)>
<Number of hyperedges/FSTs (M)>
For each terminal:
    <<Dec X-coord>> <<Dec Y-coord>> <<Hex X-coord>> <<Hex Y-coord>>
For each terminal:
    <Terminal/Steiner flag: 0=Steiner, 1=Terminal>
For each hyperedge/FST:
    <Number of terminals (Ni)>
    <Terminal indices (1..N), on one line separated by spaces>
    <Decimal length of hyperedge/FST> <Hex length of hyperedge/FST>
    <<Number of Steiner points (Mi)>>
    For each Steiner point:
       <<Dec X-coord>> <<Dec Y-coord>> <<Hex X-coord>> <<Hex Y-coord>>
    <<Number of FST edges (Ki)>>
    For each FST edge:
       <<endpoint-1>> <<endpoint-2>>
    <FST status: 0 = never needed, 1 = maybe needed, 2 = always needed>
    <Number of incompatible FSTs>
    <Incompatible FST indices (1..M), on one line separated by spaces>
\end{verbatim}
}

\newpage
\def\code#1{{\footnotesize\ttfamily #1}}
The following conventions are observed in versions 2 and 3 of the FST
data file format:
\begin{itemize}
\item 
  Data input routines require only that the individual data fields are
  separated by one or more white-space characters (space, tab, newline,
  vertical tab, and form-feed are the white-space characters of ANSI C).
\item
  Data output routines shall align items according to the schema
  above:
  \begin{itemize}
  \item Schema fields that appear on separate lines shall be written on
    separate lines.
  \item Schema fields that are all on one line shall be written all on one
    line.
  \item The data shall be indented as shown by the schema.
  \item Each indentation level shall be one ``tab stop''.
  \item The implementor may freely choose the width of this ``tab stop''.
  \end{itemize}

\item
  The \code{<Instance description (free text)>} and
  \code{<Machine description (free text)>} fields shall each be a sequence of
  0 to 79 characters.  Each character in the sequence may be any
  printable ASCII character except newline.

\item 
  The \code{<on one line separated by spaces>} fields are permitted to span
  several lines, so long as the additional lines are each indented an
  additional ``tab stop''.  The intent of this splitting is to fully pack
  lines without exceeding some column limit (e.g., 80 columns).
  If no data is to appear then the line is removed completely.

\item 
  All decimal fields shall be {\em unscaled} -- just as in the original
  terminal coordinate input data.

\item 
  The hexadecimal fields shall be {\em scaled}.  For example suppose that
  the \code{<Coordinate scaling factor>} is K.  Then the following
  relationship is implied:
  
  \centerline{\small\code{<Dec X-coord> = <Hex X-coord> / 10**K}}

  where the equal sign is meant to imply ``is within epsilon of''.
  Scaling of data shall be at the discretion of the FST generator.  For
  example the FST generator is permitted to always specify a scaling
  factor of zero -- thereby disabling the scaling feature.  Programs
  that read FST data files should not assume that the hex-values (scaled
  or otherwise) are all integral without first verifying the actual data
  values.

\item
  The \code{<Decimal Integrality delta>} (\code{<Hex Integrality
    delta>}) fields represent an {\em unscaled} ({\em scaled}) lower bound on the
  amount by which two solutions of different lengths must differ.  For
  Euclidean FSTs, this must always be 0.  For rectilinear FSTs scaled
  to integer lengths this would be 1 (scaled value).  For graphs with
  integer weights, this can also be 1.  The branch-and-cut can use
  this to provide earlier cutoff of nodes that cannot reduce the upper
  bound.

\item 
  Let fields \code{<endpoint-1>} and \code{<endpoint-2>}, occur within an FST
  containing $N$ terminals and $M$ Steiner points.  Let the field
  value be~$J$.  Then the interpretation of the endpoint field is as
  follows: 

        $1 \le J \le N \Longrightarrow$ endpoint is the $J$th terminal
                            in the FST's list of terminals.

        $-M \le J \le -1 \Longrightarrow$ endpoint is the $-J$th
                            Steiner point in the FST's list of Steiner
                            points.

\item (only applicable for version~2 of the FST data file format)\\
  Duplicate terminal groups (DTGs) identify subsets of the vertices
  having identical coordinates:
  \begin{itemize}
  \item The size of each DTG shall be at least 2.
  \item Each terminal may be listed in at most one DTG.
  \item The terminal indices listed in a single DTG must be distinct.
  \item The first terminal in each duplicate terminal group shall be
    referenced by at least one FST (having FST status $\ne$ 0).
  \item The remaining terminals in each duplicate terminal group shall NOT
      be referenced by any FST (having FST status $\ne$ 0).
    \end{itemize}
    
\item
  If an FST has ``never needed'' status then the FST generator may output
  {\em any} incompatibility and concatenation terminal information, including
  no information at all (such information is redundant).

\item
  The incompatible information shall NOT include the FST itself.

\item
  The incompatible information need not include FSTs which are ``never
  needed''.

\item 
  The incompatible information need not include FSTs which share two or
  more terminals.  It is assumed that programs that read FST data files
  are smart enough to know about such basic incompatibilities already.
  Omitting such incompatibilities can significantly reduce the size of
  the data file.

\item
  The FST-graph for rectilinear FSTs must always be a ``left-most'' and
  ``top-most'' Hwang topology.  If not, such FSTs will not appear to be
  Hwang topologies when plotted.

\item
  A simple top-down traversal of each Euclidean FST-graph starting from
  the first terminal must yield the recursive equilateral-point
  structure of the FST.  In this way, programs that read Euclidean FST
  data files are able to correctly compute the exact length of each FST
  in terms of algebraic numbers, if desired.
\end{itemize}

\printindex
\end{document}
